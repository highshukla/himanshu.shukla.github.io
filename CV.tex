%______________________________________________________________________________________________________________________
% @brief    LaTeX2e Resume for Rishabh Jha
\documentclass[margin,line]{resume}
\usepackage{xcolor}
\usepackage{amsmath}
\usepackage{verbatim}
\usepackage{graphicx}
\usepackage{hyperref}
\usepackage{amsmath}
\usepackage{amssymb}
\title{Resume_Himanshu_Shukla without homepage link}
%\usepackage{mathtools}
%______________________________________________________________________________________________________________________
\bibliographystyle{splncs}
\begin{document}
\name{\LARGE Himanshu Shukla}
\begin{resume}


    %__________________________________________________________________________________________________________________
    % Contact Information
    \section{\mysidestyle Contact\\Information}

    Master of Science.                           \hfill  Homepage: \texttt{{\href{http://www.mathe2.uni-bayreuth.de/hishukla}{http://mathe2.uni-bayreuth.de/hishukla}}}\\       
    Lehrstuhl f\"{u}r Computeralgebra, \hfill Email: \texttt{\href{mailto:Himanshu.Shukla@uni-bayreuth.de}{Himanshu.Shukla@uni-bayreuth.de}}\\ 																		%
    Mathematisches Institut,
 \\
    Universit\"{a}t Bayreuth, Germany.		
        % Research Interests
  \section{\mysidestyle Research\\Interests}

    Arithmetic Geometry (theory and computational aspects), Modular Forms, Arithmetic of Elliptic Curves. I am also interested in problems related to polynomial arithmetic, sparse polynomials and normal numbers. \vspace{-2mm}

    %__________________________________________________________________________________________________________________
    % Education
    \section{\mysidestyle Employement}
    \textbf{Chair of Computer Algebra, University of Bayreuth} \hfill \textit{ (Jan '20--Present)} \vspace{1mm}\\\vspace{1mm}%
    \textsl{Doctoral Researcher at the Chair of Computer Algebra. Deutsche Forschungsgemeinschaft (DFG) project on computation of Cassels-Tate Pairing}. \vspace{-7mm}\\\vspace{1mm}%
    
    \textbf{Max Planck Institute for Computer Science}\hfill  \textit{(Jul '19--Dec '19)} \vspace{1mm}\\\vspace{1mm}%
    \textsl{Doctoral Researcher at International Max Planck Research School for Computer Science (Algorithms and Complexity group)}.
    \vspace{-7mm}\\\vspace{1mm}%
    
    
    \section{\mysidestyle Visits}
    \textbf{Tata Institute of Fundamental Research (TIFR), Mumbai, India}\hfill  \textit{(Jun '17)} \vspace{1mm}\\\vspace{1mm}%
    \textsl{Visiting Student Research Program (VSRP-2017) candidate for reading in modular forms under Prof. N. Fakhruddin}.
   \\\vspace{-6mm}%
    
    \textbf{Indian Institute of Science Education \& Research (IISER) Pune, India} \hfill \textit{ (Dec '15)} \vspace{1mm}\\\vspace{1mm}%
    \textsl{Visiting student for reading in Algebraic Number Theory under Dr. Debargha Banerjee.} \vspace{-7mm}\\\vspace{1mm}%
    
    \section{\mysidestyle Education}
    \textbf{Chair of Computer Algebra, University of Bayreuth} \hfill \textit{ (Jan '20--Present)} \vspace{1mm}\\\vspace{1mm}%
    \textsl{PhD student. Thesis supervisor: Prof. Dr. Michael Stoll}\\\vspace{-6mm}
    
    \textbf{Max Planck Institute for Computer Science}\hfill  \textit{(Jul '19--Dec '19)} \vspace{1mm}\\\vspace{1mm}
    \textsl{PhD student. Thesis supervisor: Prof. Dr. Markus Bl\"{a}ser}\vspace{-7mm}\\\vspace{1mm}
    

    \textbf{Indian Institute of Technology (IIT) Kanpur, India} \hfill \textit{ (Jul '13 -- Jul '18)} \vspace{1mm}\\\vspace{1mm}%
    \textsl{Bachelor of Technology (B.Tech.) in Computer Science and Engineering and Master of Sciences (M.S.) in Mathematics (Dual Degree) with CPI/GPA=8.8 (PG - 9.7 \& UG - 8.6) on a scale of 10.0.} \vspace{-6mm}\\
 %__________________________________________________________________________________________________________________ 

\section{\mysidestyle Publications/ Preprints}
\begin{list2}
        

\item On Resource-Bounded versions of the van Lambalgen's theorem (joint work with Diptarka Chakr- aborty and Satyadev Nandakumar), \emph{$14^{th}$ International Conference on Theory and Applications of Models of Computation (TAMC-2017)},(\href{https://doi.org/10.1007/978-3-319-55911-7_10}{\small{\color{blue}[https://doi.org/10.1007/978-3-319-55911-7\_10]}}).\vspace{1.5mm}
\item Definable Combinatorics with Dense Linear Orders (joint work with A. Jain and A. S. Kuber) \emph{Archive for Mathematical Logic}, (\href{https://doi.org/10.1007/s00153-020-00709-8}{\small{\color{blue}[https://doi.org/10.1007/s00153-020-00709-8]}}).\vspace{1.5mm}
\item On Definable Functions of Atomless Boolean Algebras (joint work with A. S. Kuber) \emph{(in preparation)}.
\item How many zeros of random \emph{sparse} polynomials are real? (joint work with G. Jindal, A. Pandey and C. Zisopoulos), $45^{\mathrm{th}}$ \emph{International Symposium on Symbolic and Algebraic Computation (ISSAC-2020)} (\href{https://doi.org/10.1145/3373207.3404031}{\small{\color{blue}[https://doi.org/10.1145/3373207.3404031]}}).
\item Cassels-Tate Pairing on 2-selmer groups of elliptic curves. (joint work with Michael Stoll) \emph{(in prepartaion)}.
\item Computing Cassels-Tate pairing on odd-degree hyperelliptic curves. \emph{(in preparation)}.
%\item Expected number of zeros of random sparse polynomials.  (joint work with G. Jindal, A. Pandey and C. Zisopoulos) \emph{(submitted)}.
\end{list2}


    % Scholastic Achievements
    \section{\mysidestyle Awards \& \\Recognitions}
    \begin{list2}
    \item Awarded \textbf{Bhagwandas Sanghi Gold Medal} for being the best dual degree student in the Department of Mathematics and Statistics, IIT Kanpur. \hfill\textit{(Jun '18)}\vspace{.05in} 
    \item Awarded \textbf{Yogendra Nath and Sushma Gupta Scholarship} for academic performance in Computer Science and Engineering department. \hspace{2.35in}\textit{(Feb '16)}\vspace{.05in}
    \newpage
    \item Awarded \textbf{Summer Undergraduate Research Grant of Excellence (SURGE)} for summer project under Prof. Satyadev Nandakumar.\hfill\textit{(May '15 -- Jul '15)}\vspace{.05in}
    \item Awarded \textbf{Academic Excellence Award} by IIT Kanpur for achieving \textbf{10.0/10.0} GPA in first two semesters at IIT Kanpur.\hfill\textit{(Dec '14)}\vspace{.05in}
    \item Received \textbf{Dr. D. R. Bhagat Scholarship} for academic excellence at IIT Kanpur in the Computer Science and Engineering department. \hfill\textit{(Feb '14)}
    \vspace{-.1in}
    \item Secured an All India Rank of \textbf{659 (99.6 percentile)} in IIT-JEE (Advanced) 2013. \hfill\textit{(Jun '13)}\vspace{.05in}
    \item Among \textbf{top 1\% nationwide} in NSEC (National Standard Examination in \textbf{Chemistry}) and NSEA (National Standard Examination in \textbf{Astronomy}) and \textbf{top 1\% statewide} in NSEP (National Standard Examination in \textbf{Physics}) 2012.  \hspace{2.27in}\textit{(Nov '12)}
    \end{list2}
    

%\section{\mysidestyle Publications}
%\begin{itemize}
%\item Rishabh Jha, "On Origins of Sadness and Happiness", The %International Journal of Humanities \& Social Studies (THEIJHSS - ISSN: %2321-9203), Volume 4, Issue 7, July 2016, pp. 147-155. (URL: %\textit{\url{http://theijhss.com/wp-content/uploads/2016/07/22.-HS1607-056%.pdf}})
%\item Rishabh Jha, "Inconsistencies between Thermodynamics and %Relativity", International Journal of New Technologies in Science \& %Engineering (IJNTSE - ISSN: 2349-0780), Volume 3, Issue 7, July 2016. %(URL: \textit{\url{http://www.ijntse.com/upload/1468985493IJNTSE-SP-241.pd%f}})
%\item Rishabh Jha, "Kerr-Newmann Black Hole Thermodynamics, Quantum Geometry and Information Theory", International Journal of Research \& Development in Physics (IJRDP), Volume 2, Issue 1, October 2015. (URL: \textit{\url{http://www.iret.co.in/Docs/IJRDP/Volume\%202/Issue1/1.pdf}})


%\end{itemize}

    %__________________________________________________________________________________________________________________

\section{\mysidestyle Teaching \& Scribes}
\begin{list2}
        \item Tutor for the course \textbf{Einf\"{u}hrung in die Theorie die Modulformen und Modulkurven} (Introduction to the theory of modular forms and modular curves).
        \item Teaching Assistant for the course \textbf{Abstract Algebra (CS203B)}.
\end{list2}
	%Projects
\section{\mysidestyle Workshops}
\begin{list2}
        \item Workshop on \emph{Theoretical and Computational aspects of Birch and Swinnerton Dyer Conjecture} held at ICTS Banglore, India.\vspace{1.5mm}
\item Workshop on \emph{Perspectives in Complexity Theory and Cryptography} held at IISc Banglore, India.
\end{list2}


\section{\mysidestyle Talks}
\begin{list2}
        \item On Resource-Bounded versions of the van Lambalgen's theorem, \emph{14th TAMC, University of Bern}, Switzerland, 2017.\vspace{1.5mm}
        \item Model theoretic Grothendieck rings of some  structures with quantifier elimination, \emph{Math--Stat Seminar, Department of Mathematics and Statistics, IIT Kanpur}, 2018.\vspace{1.5mm}
        \item On expected number of zeros of a random sparse polynomial, \emph{Graduate Research Seminar}, Max Planck Institute for Computer Science, Saarbr\"{u}cken, 2019.\vspace{1.5mm}
        \item Cassels-Tate pairing on elliptic curves, 
        \emph{Oberseminar Arithmetische Geometrie}, Universit\"{a}t Bayreuth, 2020.
         \vspace{1.5mm}
        \item Playing dodgeball with normality,
        \emph{Oberseminar Arithmetische Geometrie}, Universit\"{a}t Bayreuth, 2021.\vspace{1.5mm}
        \item Cassels-Tate pairing on hyperelliptic curves,
        \emph{Oberseminar Arithmetische Geometrie}, Universit\"{a}t Bayreuth, 2021.
        \item Computing Cassels-Tate pairing on odd-degree hyperelliptic curves,
        \emph{Oberseminar Arithmetische Geometrie}, Universit\"{a}t Bayreuth, 2022.
	\item Computing Cassels-Tate pairing on the 2-Selmer group of odd-degree hyperelliptic curves.
		\emph{Rational Points}, Schloss Schney, 2022.
\end{list2}
\begin{comment}
	\section{\mysidestyle Research Projects \&\\Experiences}

\textbf{Modular Forms and Elliptic Curves \href{http://home.iitk.ac.in/~hishukla/Modular_forms_part_1.pdf}{\small{\color{blue}[PDF]}},  \href{http://home.iitk.ac.in/~hishukla/Ramanujan_tau.pdf}{\small{\color{blue}[PDF]}}}\\
\textsl{\footnotesize Under Dr. Somnath Jha, IIT Kanpur}\hfill\textit{(Aug '17 -- Dec '17)}\vspace{-3mm}\\ 
\begin{list2}
    \item Read on connections between Modular Forms and Elliptic Curves.\vspace{1,5mm}
    \item Read initial chapters of the book \emph{``Arithmetic of Elliptic Curves"} by J.H. Silverman.
\end{list2}
\vspace{-2mm}
\textbf{Cantor-Zassenhaus type algorithm for polynomial factoring over finite fields \href{http://home.iitk.ac.in/~hishukla/CZ_type_algo.pdf}{\small{\color{blue}[PDF]}}}\\ 
    \textsl{\footnotesize Under Dr. Nitin Saxena \& Dr. Rajat Mittal, IIT Kanpur}\hfill\textit{(Dec '15 -- Apr '16)}\vspace{-3mm}\\ 
    \begin{list2}
\item Explored a \emph{Cantor-Zassenhaus} type algorithm for factoring polynomials over finite fields. \vspace{1.5mm}
\item The algorithm assuming \emph{Generalized Reimann's Hypothesis} factors polynomials.  
\end{list2}
    \vspace{-2mm}
    
    \textbf{Generalized form of Burgess' Lemma 2 and easier proof of deterministic bound on polynomial factoring over finite fields \href{http://home.iitk.ac.in/~hishukla/factoring.pdf}{\small{\color{blue}[PDF]}}} \hfill\textit{(Jul '15 -- Nov '15)} \vspace{-0mm}\\\vspace{1mm}
    \textsl{\footnotesize Under Dr. Nitin Saxena \& Dr. Rajat Mittal, IIT Kanpur}\vspace{-3mm}\\
    \begin{list2}
\item Studied \emph{Burgess' inequality} and extended \emph{Burgess' Lemma 2} to arbitrary degree polynomials. \vspace{1.5mm}
\item Conducted experiments to check the distribution of quadratic residues and non-residues over $\mathbb{F}_p$.
\end{list2}
   \vspace{-2mm}
    
    \textbf{Analogues of Miller-Yu theorem in resource bounded measures
 \href{http://home.iitk.ac.in/~hishukla/miller-yu.pdf}{\small{\color{blue}[PDF]}}}		 \vspace{0mm}\\\vspace{1mm}%
    \textsl{\footnotesize Under Dr. Satyadev Nandakumar, IIT Kanpur (SURGE- 2015 Project)}\hfill\textit{(May '15 -- Jul '15)}\vspace{-3mm}\\
    \begin{list2}
\item Studied different randomness paradigms and Martin L\"of randomness.\vspace{1.5mm}
\item Studied Miller-Yu theorem and its analogues in resource bounded measures and derived resource bounded versions of the Chaitin's inequality.
\end{list2}
\end{comment}
    %----------------------------------------------------------------------------------------------------------_________


 %______________________________________________________________________________________________________________________
%\section{\mysidestyle Leadership Skills }
% \textbf{Cheif Technical Advisor at Atventus Technologies}		\hfill					\textit{(Nov '16 -- May '18)} \\\vspace{-4mm}% 
 %   \begin{list2}
%\item Among one of the founding members of the company and responsible for handelling all major technical issues of the company.
%\end{list2}
 %   \vspace{-3mm}%}    

 %______________________________________________________________________________________________________________________	
 
    %__________________________________________________________________________________________________________________
    % Computer Skills
    \section{\mysidestyle Technical\\Skills} 
    
    
    \textbf{Programming Languages \& Software} - C/C++, Python, MATLAB, Octave, Sage, Magma, \LaTeX.\\


	%_____________________________________________________________________________________________________________	
	
    
%______________________________________________________________________________________________________________________
\end{resume}
\bibliography{citation}




\end{document}


%______________________________________________________________________________________________________________________
% EOF

